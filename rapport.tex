\documentclass{report}
\usepackage[T1]{fontenc}
\usepackage[utf8]{inputenc}
\usepackage[francais]{babel}
\usepackage{graphicx}
\usepackage{verbatim}
\usepackage{moreverb}
\usepackage{hyperref}


\title{\textbf{Rapport Projet de stage : Limitedness}}
\author{
\bsc {Brebant} Alexandre\\
\bsc {Xue} Juedong}

\begin{document}
\maketitle

\tableofcontents
\chapter*{Introduction au problème}
Durant ce stage, nous nous sommes interessé aux automates à distance (distance automata). Ce sont des automates finis, non déterministes, auxquels on associe des coûts pour chaque transition. On associe donc également un coût pour chaque mot accepté par l'automate. Le coût d'un mot m correspond au minimum des coûts de tous les chemins de l'automate reconnaissant m.\\
Le problème de limitedness est de savoir si il existe une majoration des coûts des mots accepté par un automate à distance.\\
Ce type d'automate et son problème de limitedness a été présenté par Hashiguchi dans CITATION pour la résolution d'une autre célèbre problématique, la question de hauteur d'étoile (star height problem). Le travail d'Hashiguchi étant considéré comme difficile à comprendre, d'autre chercheurs, comme Simon \cite{Simon:Semigroups-Matrices-over-Tropical:1994:a} ou Leung \cite{Leung&Podolskiy:limitedness-problem-distance-automata::2004:a}, ont étudié le problème et apporté d'autres solutions sur le sujet.\\\\
Notre travail a donc consisté, dans un premier temps, à lire et comprendre les différents travaux de ces chercheurs afin d'être capable d'implémenter un algorithme permettant la détection de limite dans un automate à distance.

\part{Mise en place théorique de l'algorithme}

Le papier dont nous nous somme le plus servit afin de comprendre et résoudre le problème de limitedness, est \cite{Simon:Semigroups-Matrices-over-Tropical:1994:a} mais il nous a d'abord fallu comprendre la complexité du problème.

\chapter{De l'automate aux matrices}

\chapter{Matrices idempotentes}

\chapter{Matrices stables}

\part{Exécution détaillée du programme}

\chapter{Réutilisation du programme d'Adrien Boussicault}

(présentation des différents modules et fonctions disponible + détails des ajouts pour la gestion des cout, de l'affichage et autres fonctions additionnelles que nous avons ajouté)

\chapter{Création et manipulation des matrice}

(creer\_matrice\_transition, multiplication, etc...)

\chapter{Structure de stockage des matrices}

(mautomate, descrption des fonctions de création)

\chapter{Résolution du problème de limitedness}

\bibliographystyle{abbrv}
\bibliography{limitedness}

\end{document}
